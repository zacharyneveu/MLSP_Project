The conclusions of this work are that \ac{KSVD} is capable of extracting notes from piano music in the time domain. This learning method can include a parameter specifying the maximum number of non-zero notes per frame, as well as the maximum number of possible notes to learn. These features make the technique generalizable to a large variety of instruments. One continuation of this work that could further improve performance would be to learn a complex, rather than real, vector $Z$. In the current scheme with a real value of $Z_i$, atom $D_i$ contributes best only when it has a phase of either \ang{0} or \ang{180}. Making $Z_i$ complex would allow $D_i$ to contribute more when $X$ is out of phase from the other training data. Another continuation of this work would be to pair the current algorithm with a note onset detection algorithm and measure the performance transcribing when there are some errors in the frame boundaries. Under these conditions, it would be possible to determine the extent to which this algorithm is feasible for real-world tasks.
