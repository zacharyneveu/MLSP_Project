For the purposes of this paper, it is assumed that a method for onset detection such as the one presented in \cite{bock_evaluating_nodate} is available and performs well.  The input for the problem is then time-domain frames of music each of which contains at least one note and at least one onset. The goal is to factor each frame $F$ into an approximate linear combination $Z$ of a dictionary of notes $\mathbf{D}$ as expressed by \eqref{eq:factor}. The dimensionality of the variables are $F,\ Z \in R^{N\times n}$ and $D\in R^{n\times K}$ where $N$ is the number of audio frames, $n$ is the length of a frame, and $K$ is the number of atoms in the dictionary. The constraint is imposed that the number of non-zero elements in $Z \le k$ where $k \ll K$. For our specific purposes $K$ was chosen to be $88$ because a piano has $88$ possible notes, and $k$ was chosen to be $10$ as the average pianist has $10$ fingers.

\begin{equation}
\begin{tabular}{cr}
	$F \approx \mathbf{D}Z$ & $s.t. \|Z\|_0 \le k$
\end{tabular}
\label{eq:factor}
\end{equation}
